\documentclass[twoside,11pt]{article}

% Any additional packages needed should be included after jmlr2e.
% Note that jmlr2e.sty includes epsfig, amssymb, natbib and graphicx,
% and defines many common macros, such as 'proof' and 'example'.
%
% It also sets the bibliographystyle to plainnat; for more information on
% natbib citation styles, see the natbib documentation, a copy of which
% is archived at http://www.jmlr.org/format/natbib.pdf

% Available options for package jmlr2e are:
%
%   - abbrvbib : use abbrvnat for the bibliography style
%   - nohyperref : do not load the hyperref package
%   - preprint : remove JMLR specific information from the template,
%         useful for example for posting to preprint servers.
%
% Example of using the package with custom options:
%
% \usepackage[abbrvbib, preprint]{jmlr2e}

\usepackage{jmlr2e}
\usepackage{minted}
\usepackage{textcomp}
\usepackage{graphicx}
\usepackage{amsmath}
\usepackage{bm}
\usepackage{booktabs}      % for \toprule in tables
\usepackage{multirow}      % for table multirow
\usepackage{tcolorbox}   % colorbox in minted
\usepackage{parcolumns}
\usepackage{adjustbox}
\usepackage{nicefrac}
\usepackage{tabularx}
\usepackage{array}
\usepackage{wasysym}

% Definitions of handy macros can go here

\newcommand{\dataset}{{\cal D}}
\newcommand{\fracpartial}[2]{\frac{\partial #1}{\partial  #2}}

\newcolumntype{R}[2]{%
  >{\adjustbox{angle=#1,lap=\width-(#2)}\bgroup}%
  l%
  <{\egroup}%
}
\newcommand*\rot{\multicolumn{1}{R{30}{2.0em}}}

% Heading arguments are {volume}{year}{pages}{date submitted}{date published}{paper id}{author-full-names}
\jmlrheading{VV}{YYYY}{page-page}{X/YY}{X/YY}{TODO}{Ryan Curtin, Marcus Edel, Rahul Ganesh Prabhu, Suryoday Basak, Zhihao Lou, Conrad Sanderson}

% Short headings should be running head and authors last names
\ShortHeadings{The ensmallen library for flexible numerical optimization}{Curtin, Edel, Prabhu, Basak, Lou and Sanderson}

\firstpageno{1}

\begin{document}

\title{The ensmallen library for flexible numerical optimization }

% compact format for list of authors and affiliations
\author{%
  \name Ryan R. Curtin \hfill \email ryan@ratml.org ~|~ \addr RelationalAI, Atlanta, GA 30318, USA\\
  \name Marcus Edel \hfill \addr Free University of Berlin, Germany\\
  \name Rahul Ganesh Prabhu \hfill \addr Birla Institute of Technology and Science Pilani, India\\
  \name Suryoday Basak \hfill \addr University of Texas at Arlington, USA\\
  \name Zhihao Lou \hfill \addr Epsilon, Chicago, IL, USA\\
  \name Conrad Sanderson \hfill \addr Data61/CSIRO, Australia, and Griffith University, Australia%
  }

% %% super space wasting format for list of authors and affiliations
% \author{%
%   \name Ryan R. Curtin \email ryan@ratml.org \\
%   \addr RelationalAI, Atlanta, GA 30318, USA
%   \AND
%   \name Marcus Edel\\
%   \addr Free University of Berlin, Germany
%   \AND
%   \name Rahul Ganesh Prabhu \\
%   \addr Birla Institute of Technology and Science Pilani, India
%   \AND
%   \name Suryoday Basak \\
%   \addr University of Texas at Arlington, USA
%   \AND
%   \name Zhihao Lou \\
%   \addr Epsilon, Chicago, IL, USA
%   \AND
%   \name Conrad Sanderson \\  %% NOTE: ** DO NOT ** LIST MY EMAIL ADDRESS; I DON'T WANT SPAM OR JOB OFFERS FROM AMAZON
%   \addr Data61/CSIRO, Australia, and Griffith University, Australia%
%   }

\editor{not currently known}

\maketitle

\begin{abstract}%   <- trailing '%' for backward compatibility of .sty file
We overview the {\tt ensmallen} numerical optimization library,
which provides a flexible C++ framework
for mathematical optimization of user-supplied objective functions.
Many types of objective functions are supported,
including general, differentiable, separable, constrained, and categorical.
A~diverse set of pre-built optimizers is provided,
including Quasi-Newton optimizers and many variants of Stochastic Gradient Descent.
The underlying framework facilitates the implementation of new optimizers.
Optimization of an objective function typically requires supplying only one or two {C++} functions.
Custom behavior during optimization can be easily specified via callback functions.
Empirical comparisons show that {\tt ensmallen}
can outperform other optimization frameworks while providing more functionality.
The library is available at \url{https://ensmallen.org}
and is distributed under the permissive BSD license.

\end{abstract}

\begin{keywords}
  Numerical optimization, mathematical optimization, function minimization.
\end{keywords}


\section{Introduction}

The problem of numerical optimization is generally expressed as
$\operatornamewithlimits{argmin}_x f(x)$
where $f(x)$ is a given objective function and $x$ is typically a vector or matrix.
Such optimization problems are fundamental and ubiquitous in the computational sciences~\citep{Nocedal_2006}.
Many frameworks or libraries for specific machine learning approaches
have an integrated optimization component for distinct and limited use cases,
such as
TensorFlow~\citep{TensorFlow_arXiv_2016},
PyTorch~\citep{PyTorch_NeurIPS_2019}
and LibSVM~\citep{libsvm2011}.
There are also many general numerical optimization toolkits
aimed at supporting a wider range of use cases,
including SciPy~\citep{SciPy_arXiv_2019},
opt++~\citep{meza1994opt++},
and 
OR-Tools~\citep{ortools} among many others.
% CVXOPT~\citep{vandenberghe2010cvxopt},
%NLopt~\citep{johnson2014nlopt}, Ceres~\citep{ceres-solver},
%% {\tt fminsearch()} in MATLAB~\cite{matlab_fminsearch},
%and RBFOpt~\citep{costa2018rbfopt}.
However, such toolkits still have limitations in several areas,
including:
(i)~types of supported objective functions,
(ii)~selection of available optimizers,
(iii)~support for custom behavior via callback functions,
(iv)~support for various underlying element and matrix types used by objective functions.
and
(v)~extensibility, to facilitate adding more optimizers.

The abovementioned shortcomings have motivated us to implement the {\tt ensmallen} library,
which explicitly supports numerous types of user-defined objective functions,
including general, differentiable, separable, categorical, constrained, and semidefinite.
Custom behavior during optimization can be specified via {callback} functions,
for purposes such as printing progress, early stopping, inspection and modification of an optimizer's state,
and debugging of new optimizers.
A~large and diverse set of pre-built optimizers is provided;
at the time of writing, 46 optimizers are available.
This includes 
simulated annealing \citep{kirkpatrick1983optimization},
several Quasi-Newton optimizers \citep{liu1989limited,mokhtari2018},
and many variants of Stochastic Gradient Descent \citep{Ruder_2016}.
The user-facing interface to the optimizers is intuitive
and matches the ease of use of popular
optimization toolkits mentioned above;
see the online documentation at \mbox{\url{https://ensmallen.org/docs.html}} for more details.
Furthermore, {\tt ensmallen} supports the use of various underlying element and matrix types.
This includes single- and double-precision floating point values~\citep{Goldberg_CSUR_1991}, 
integer values, and data compactly stored as sparse matrices.
Lastly, {\tt ensmallen} provides a framework to easily allow the implementation of new optimization techniques.
Table~\ref{tab:comparison} compares the functionality provided
by {\tt ensmallen} and other optimization toolkits.


\begin{table}[!t]
\footnotesize
\centering
    \begin{tabular}{@{} cl*{8}c @{}}
%  \begin{tabular}{ccccccc}
          & 
          & \multicolumn{6}{c}{} \\[0.6ex]
          & 
            %
            % If I can easily implement any new optimization technique to use
          & \rot{\scriptsize easily extensible}
            %
            % If there is any support for constrained optimization
          & \rot{\scriptsize constraints}
            %
            % If the optimization framework can do mini-batch
          & \rot{\scriptsize batches}  % NOTE: originally as: separable fns.~/~batches
            % 
            % If I can implement any arbitrary (general) function to be optimized
          & \rot{\scriptsize general objective fns.}
            %
            % the framework could take advantage of when the gradient is sparse
          & \rot{\scriptsize sparse gradients}
            %
            % the framework can handle categorical/discrete variables
          & \rot{\scriptsize categorical variables}
            %
            % If any type can be optimized, this is true
            & \rot{\scriptsize various element types}
            %
            % If callback support is available.
          & \rot{\scriptsize callback fns.} \\
        \cmidrule[1pt]{2-10}
        % 
        % It might be reasonable to say ensmallen categorical support is only partial,
        % but I am not sure exactly where we draw the line.
        & \texttt{ensmallen}
        & \CIRCLE & \CIRCLE & \CIRCLE & \CIRCLE & \CIRCLE & \CIRCLE & \CIRCLE & \CIRCLE\\
        %
        % The Shogun toolbox has a fairly nice framework, but it doesn't support
        % sparse gradients or categorical features.  It also does not appear to
        % support constraints, arbitrary types, or callbacks.
        & Shogun \citep{sonnenburg2010shogun}
        & \CIRCLE & - & \CIRCLE & \CIRCLE & - & - & - & - \\
        % 
        % VW doesn't appear to have any framework whatsoever and the code is
        % awful, but it does support batches and categorical features.
        & VW \citep{Langford2007VW}
        & - & - & \CIRCLE  & - & - & \CIRCLE & - & - \\
        % 
        % TensorFlow has a few optimizers, but they are all SGD-related.  You
        % can write most objectives easily (but some very hard), and categorical
        % support might be possible but would not be easy.
        & TensorFlow \citep{TensorFlow_arXiv_2016}
        & - & -  & \CIRCLE  & \LEFTcircle & \LEFTcircle & - & \LEFTcircle & - \\
        % 
        % PyTorch is increasingly popular these days.  It has Caffe integrated
        % into itself, but this refers to the actual PyTorch optimizer parts.
        & PyTorch \citep{PyTorch_NeurIPS_2019}
        & \LEFTcircle & - & \CIRCLE & \LEFTcircle & - & - & \LEFTcircle & - \\
        % 
        %% Caffe has a nice framework, but it's only for SGD-related optimizers.
        %% I think I could write a new one, but it is not the easiest thing in
        %% the world.
        %% & Caffe \citep{jia2014caffe}
        %% & \LEFTcircle & \CIRCLE & -  & \CIRCLE & \LEFTcircle & - & - & \LEFTcircle & \CIRCLE \\
        %
        % Keras is restricted to neural networks and SGD-like optimizers.
        % I don't know that it is possible to easily write a new optimizer.
        & Keras \citep{chollet2015keras}
        & \LEFTcircle & -  & \CIRCLE & \LEFTcircle & - & - & \LEFTcircle & \CIRCLE \\
        % 
        % sklearn has a few optimizer frameworks, but they are all in different
        % places and have somewhat different support.
        & scikit-learn \citep{pedregosa2011scikit}
        & - & - & \LEFTcircle  & \LEFTcircle & - & - & \LEFTcircle & - \\
        % 
        % scipy has some nice optimizer framework but it does not support
        % batches or some of the more complex functionality.  And you can't
        % write your own.
        & SciPy \citep{SciPy_arXiv_2019}
        & - & \CIRCLE  & -  & \CIRCLE & - & - & \LEFTcircle & \CIRCLE \\
        % 
        % MATLAB is very similar to scipy.
        & MATLAB \citep{mathworks2017OTB}
        & - & \CIRCLE & - & \CIRCLE & - & - & \LEFTcircle & - \\
        % 
        % Optim.jl isn't the only Julia package for optimization, but it's the
        % one we compare against.
        % & \texttt{Optim.jl} \citep{mogensen2018optim}   &
        & \texttt{Optim.jl} (Mogensen et al.~2018) \nocite{mogensen2018optim}
        & - & \LEFTcircle & - & \CIRCLE & - & - & \CIRCLE & \CIRCLE \\
        \cmidrule[1pt]{2-10}
    \end{tabular}
\vspace*{-0.5em}
\caption{
Feature comparison:
\CIRCLE~= available,
\LEFTcircle~= partially available,
-~= not available.
\label{tab:comparison}
\vspace{-1.5ex}
}
\end{table}


\section{Functionality}

The task of optimizing an objective function with {\tt ensmallen} is straightforward.
The type of objective function defines the implementation requirements.
Each type has a minimal set of methods that must be implemented,
which typically is only between one and four methods.
Apart from the baseline requirement of an implementation of $f(x)$,
characteristics of $f(x)$ can be exploited through additional functions.
For example, if $f(x)$ is differentiable,
an implementation of $f'(x)$ can be used to speed up the optimization process.
One of the pre-built optimizers for differentiable functions,
such as L-BFGS~\citep{liu1989limited},
can be immediately employed.

Whenever possible, {\tt ensmallen} will automatically infer methods
that are not provided.
For example, given a separable objective function
$f(x) = \sum_i f_i(x)$
where an implementation of $f_i(x)$ is provided
(as well as the number of such separable objectives),
an implementation of $f(x)$ can be automatically inferred.
This is done at compile-time, and so there is no additional runtime
overhead compared to a manual implementation.
C++ template metaprogramming techniques~\citep{Abrahams_2004,alexandrescu2001modern}
are internally used to automatically produce efficient code during compilation.

To implement a new optimizer,
only one function needs to be defined inside a class,
which can then expect an external implementation of $f(x)$ to be available.
If the optimizer is restricted to handling differentiable functions,
it can also expect an implementation of $f'(x)$ to be available.

When an optimizer (either pre-built or new) is used with a user-provided objective function,
an internal mechanism automatically checks the requirements for that optimizer
(eg., presence of an implementation of $f'(x)$),
resulting in user-friendly error messages at compile-time
if any required methods are not detected.
For example, as L-BFGS is suited for differentiable functions,
a compile-time error will be printed if an attempt is made
to use it with non-differentiable (general) functions.

%% SIMPLIFIED EXAMPLE CODE
%% 
\begin{figure}[b!]
\hrule
\vspace{1ex}
\centering
\begin{minted}[fontsize=\footnotesize]{c++}

#include <ensmallen.hpp>

struct LinearRegressionFn
{
  LinearRegressionFn(const arma::mat& in_X, const arma::vec& in_Y) : X(in_X), y(in_Y) {}

  double Evaluate(const arma::mat& phi)
    { const arma::vec tmp = X * phi - y;  return arma::dot(tmp,tmp); }
  
  void Gradient(const arma::mat& phi, arma::mat& grad)
    { grad = 2 * X.t() * (X * phi - y); }

  const arma::mat& X; const arma::vec& y;
};

int main() 
{
  arma::mat X; arma::vec y;
  // ... set the contents of X and y here ...
  arma::mat phi_star(X.n_rows, 1, arma::fill::randu);  // initial point (uniform random)
  LinearRegressionFn f(X, y);
  ens::L_BFGS optimizer; // create an optimizer object with default parameters
  optimizer.Optimize(f, phi_star);
  // at this point phi_star contains the optimized parameters
}
\end{minted}
\hrule
\vspace*{-0.5em}
\caption{Example implementation of an objective function class for linear
regression and usage of the L-BFGS optimizer.
The optimizer can be easily changed by replacing
{\tt ens::L\_BFGS} with another optimizer,
such as {\tt ens::GradientDescent},
or {\tt ens::SA} which implements simulated annealing \citep{kirkpatrick1983optimization}.
%The online documentation for all ensmallen optimizers
%is at \mbox{\url{https://ensmallen.org/docs.html}}.
%The {\tt arma::mat} and {arma::vec} types are
%dense matrix and vector classes
%from the Armadillo linear algebra library~\cite{sanderson2016armadillo},
%with the corresponding online documentation at
%\mbox{\url{http://arma.sf.net/docs.html}}
%.
}
\label{fig:lr_function}
\vspace*{-2em}
\end{figure}


\section{Example Usage \& Empirical Comparison}

For demonstration purposes, let us consider the problem of linear regression.
A matrix of predictors $\bm X \in \mathcal{R}^{d \times n}$
and a vector of responses $\bm y \in \mathcal{R}^n$ is given.
The task is to find the best linear model $\bm \Phi \in \mathcal{R}^d$,
which translates to finding
$\bm \Phi^* = \operatornamewithlimits{argmin}_{\bm\Phi} f(\bm \Phi)$ for
%$f(\bm \Phi) = \| \bm X \bm \Phi - \bm y \|^2 = (\bm X \bm \Phi - \bm y)^{\top} (\bm X \bm \Phi - \bm y).$
$f(\bm \Phi) = \| \bm X \bm \Phi - \bm y \|^2.$
From this we can derive the gradient
$f'(\bm \Phi) = 2 \bm X^{\top} (\bm X \bm \Phi - \bm y).$

To find $\bm \Phi^*$ using a differentiable optimizer,
we simply need to provide implementations of $f(\bm \Phi)$ and $f'(\bm \Phi)$.
For a differentiable function, {\tt ensmallen} requires only two methods:
{\tt Evaluate()} and {\tt Gradient()}.
The pre-built L-BFGS optimizer can then be used to find~$\bm \Phi^*$.
Figure~\ref{fig:lr_function} shows an example implementation.
Via the use of the Armadillo library~\citep{sanderson2016armadillo},
the linear algebra expressions to implement the objective function and its gradient
are compact and closely match natural mathematical notation.
Armadillo efficiently translates the expressions into standard BLAS and LAPACK function calls~\citep{anderson1999lapack},
allowing easy exploitation of high-performance implementations such as the multi-threaded \mbox{OpenBLAS}~\citep{OpenBLAS} and Intel MKL~\citep{IntelMKL} libraries.

Table~\ref{tab:lbfgs} compares the performance
of {\tt ensmallen} against other optimization frameworks
for the task of optimizing linear regression parameters on various dataset sizes.
We use the {\tt bfgsmin()} function from GNU Octave \citep{octave}.
We also use automatic differentiation for Julia via {\tt ForwardDiff.jl} \citep{RevelsLubinPapamarkou2016}
and for Python via Autograd \citep{maclaurin2015autograd}.
In each framework the provided L-BFGS optimizer is explicitly limited to $10$ iterations.
The data used are highly noisy random data with a slight linear pattern.
The runtimes are the average of 10 runs.
The experiments were done on a MacBook Pro i7 (2018 model),
with clang 1000.10.44.2, Julia version 1.0.1, Python 2.7.15, and Octave 4.4.1.
The results show that {\tt ensmallen} obtains the lowest run times.

%% RESULTS WITH ONLY ONE ROW FOR ENSMALLEN
%% 
\begin{table}[t!]
{\small
\centering
%\begin{adjustbox}{scale={0.90}{0.90}}
\begin{tabular}{lccccc}
\toprule
{\em Framework} & $d$: 100, $n$: 1k & $d$: 100, $n$: 10k & $d$: 100, $n$:
100k & $d$: 1k, $n$: 100k \\
\midrule
\texttt{ensmallen} & {\bf 0.001s} & {\bf 0.009s} & {\bf 0.154s} & {\bf 2.215s} \\
% Dropped for space and awful performance
%\texttt{Calculus.jl} & 0.172s & 0.960s & 27.428s & 2535.507s \\
\texttt{Optim.jl}  & 0.006s & 0.030s & 0.337s & 4.271s \\
\texttt{SciPy} & 0.003s & 0.017s & 0.202s & 2.729s \\
\texttt{bfgsmin()} & 0.071s & 0.859s & 23.220s & 2859.81s\\
% It's possible to tune ForwardDiff.jl a bit, but it doesn't give significant
% speedups to make it competitive and it really makes the code ugly.
\texttt{ForwardDiff.jl} & 0.497s & 1.159s & 4.996s & 603.106s \\
\texttt{Autograd} & 0.007s & 0.026s & 0.210s & 2.673s \\
\bottomrule
\end{tabular}
%\end{adjustbox}
%\vspace*{0.25ex}
\vspace*{-0.4em}
\caption{
Runtimes for optimizing linear regression parameters on various dataset sizes,
where $n$ is the number of samples,
and $d$ is the dimensionality of each sample.
%All Julia runs do not count compilation time.
}
\label{tab:lbfgs}
}
\vspace*{-2.2em}
\end{table}


\section{Conclusion}

The {\tt ensmallen} numerical optimization provides a flexible framework
for optimization of user-supplied objective functions in C++.
Unlike other frameworks, {\tt ensmallen} supports many types of objective functions,
provides a diverse set of pre-built optimizers,
supports custom behavior via callback functions,
and handles various element and matrix types used by objective functions.
The underlying framework facilitates the implementation of new optimization techniques,
which can be contributed for inclusion into the library.

The library has been successfully used by open source projects
such as the {\it mlpack} machine learning toolkit~\citep{mlpack2018}.
The library uses the permissive BSD license~\citep{Laurent_2008},
with the development done in an open and collaborative manner.
The source code and documentation are freely available at \mbox{\url{https://ensmallen.org}}.

Further details, such as internal use of template metaprogramming
for automatic generation of efficient code, automatic function inference,
clean error reporting, and various approaches for obtaining efficiency
are discussed in the accompanying technical report~\citep{ensmallen2020}.


% Acknowledgements should go at the end, before appendices and references

\bibliography{refs}

\end{document}
