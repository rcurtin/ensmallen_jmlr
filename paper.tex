\documentclass[twoside,11pt]{article}

% Any additional packages needed should be included after jmlr2e.
% Note that jmlr2e.sty includes epsfig, amssymb, natbib and graphicx,
% and defines many common macros, such as 'proof' and 'example'.
%
% It also sets the bibliographystyle to plainnat; for more information on
% natbib citation styles, see the natbib documentation, a copy of which
% is archived at http://www.jmlr.org/format/natbib.pdf

% Available options for package jmlr2e are:
%
%   - abbrvbib : use abbrvnat for the bibliography style
%   - nohyperref : do not load the hyperref package
%   - preprint : remove JMLR specific information from the template,
%         useful for example for posting to preprint servers.
%
% Example of using the package with custom options:
%
% \usepackage[abbrvbib, preprint]{jmlr2e}

\usepackage{jmlr2e}
\usepackage{minted}
\usepackage{textcomp}
\usepackage{graphicx}
\usepackage{amsmath}
\usepackage{bm}
\usepackage{booktabs}      % for \toprule in tables
\usepackage{multirow}      % for table multirow
\usepackage{tcolorbox}   % colorbox in minted
\usepackage{parcolumns}
\usepackage{adjustbox}
\usepackage{nicefrac}
\usepackage{tabularx}
\usepackage{array}
\usepackage{wasysym}

\renewcommand{\baselinestretch}{1.1}\small\normalsize

\newcolumntype{R}[2]{%
  >{\adjustbox{angle=#1,lap=\width-(#2)}\bgroup}%
  l%
  <{\egroup}%
}
\newcommand*\rot{\multicolumn{1}{R{30}{2.0em}}}

% Definitions of handy macros can go here

\newcommand{\dataset}{{\cal D}}
\newcommand{\fracpartial}[2]{\frac{\partial #1}{\partial  #2}}

% Heading arguments are {volume}{year}{pages}{date submitted}{date published}{paper id}{author-full-names}
\jmlrheading{VV}{YYYY}{page-page}{X/YY}{X/YY}{TODO}{Ryan Curtin et al}

% Short headings should be running head and authors last names
\ShortHeadings{The ensmallen library for flexible numerical optimization}{Curtin, Edel, Prabhu, Basak, Lou and Sanderson}

\firstpageno{1}

\begin{document}

\title{The ensmallen library for flexible numerical optimization}

\author{\name Ryan R. Curtin \email ryan@ratml.org \\
       \addr RelationalAI, Atlanta, GA 30318, USA
       \AND
       \name Marcus Edel\\
       \addr Free University of Berlin, Germany
       \AND
       \name Rahul Ganesh Prabhu \\
       \addr Birla Institute of Technology and Science Pilani, India
       \AND
       \name Suryoday Basak \\
       \addr University of Texas at Arlington, USA
       \AND
       \name Zhihao Lou \\
       \addr Epsilon, Chicago, IL, USA
       \AND
       \name Conrad Sanderson \\
       \addr Data61/CSIRO, Australia, and Griffith University, Australia}

\editor{not currently known}

\maketitle

\begin{abstract}%   <- trailing '%' for backward compatibility of .sty file
We overview the {\tt ensmallen} numerical optimization library,
which provides a flexible C++ framework
for mathematical optimization of user-supplied objective functions.
Many types of objective functions are supported,
including general, differentiable, separable, constrained, and categorical.
A large set of pre-built optimizers is provided,
including many variants of Stochastic Gradient Descent and Quasi-Newton optimizers.
Implementation of a new optimizer requires only one function,
while a new objective function typically requires only one or two C++ functions.
Custom behavior during optimization can be easily specified via callback functions.
Empirical comparisons show that {\tt ensmallen}
can outperform other optimization frameworks (such as Julia and SciPy)
while providing more flexibility.
The library is available at \url{https://ensmallen.org}
and is distributed under the permissive BSD license.

\end{abstract}

\begin{keywords}
  Numerical optimization, mathematical optimization, function minimization.
\end{keywords}

\section{Introduction}
\label{sec:introduction}

The problem of numerical optimization is in short expressed as
$\operatornamewithlimits{argmin}_x f(x)$
where $f(x)$ is a given objective function and $x$ is typically a vector or matrix.
Such optimization problems are fundamental and ubiquitous in the computational sciences~\citep{Nocedal_2006}.
Many frameworks or libraries for specific machine learning approaches
have an integrated optimization component,
such as
TensorFlow~\citep{tensorflow2015-whitepaper},
PyTorch~\citep{NEURIPS2019_9015}
and LibSVM~\citep{libsvm2011}.
There are also many general numerical optimization toolkits,
including SciPy~\citep{2019arXiv190710121V},
opt++~\citep{meza1994opt++},
and 
OR-Tools~\citep{ortools} among many others.
% CVXOPT~\citep{vandenberghe2010cvxopt},
%NLopt~\citep{johnson2014nlopt}, Ceres~\citep{ceres-solver},
%% {\tt fminsearch()} in MATLAB~\cite{matlab_fminsearch},
%and RBFOpt~\citep{costa2018rbfopt}.
However, such toolkits have limitations in several areas,
including:
(i)~types of supported objective functions,
(ii)~selection of available optimizers,
(iii)~support for custom behavior via callback functions,
and
(iv)~support for various element types for storage.

% none of them is flexible enough to support a wide range of use cases,

The abovementioned shortcomings have motivated us to implement the {\tt ensmallen} library,
which explicitly supports numerous types of user-defined objective functions,
including general, differentiable, separable, categorical, constrained, and semidefinite.
Custom behavior during optimization can be specified via {callback} functions,
for purposes such as printing progress, early stopping, or inspection and modification of an optimizer's state.
A large and intuitive set of pre-built optimizers is also provided;
at the time of writing, 46 optimizers are available.
This includes many variants of Stochastic Gradient Descent and Quasi-Newton optimizers.
The external interface to the optimizers is intuitive
and matches the ease of use of popular
optimization toolkits mentioned above.
For details see the online documentation at \mbox{\url{https://ensmallen.org/docs.html}}.
Furthermore, {\tt ensmallen} supports 
Table~\ref{tab:comparison} compares functionality provided
by {\tt ensmallen} and other optimization toolkits.


\begin{table}[!t]
\footnotesize
\centering
    \begin{tabular}{@{} cl*{9}c @{}}
%  \begin{tabular}{ccccccc}
        & & \multicolumn{7}{c}{} \\[0.6ex]
            %
            % If there is any coherent framework at all, this is true.
        & & \rot{unified framework}
            %
            % If there is any support for constrained optimization, this is
            % true.
          & \rot{constraints}
            %
            % If the optimization framework can do mini-batch, this is true.
          & \rot{separable functions / batches}
            % 
            % If I can implement any arbitrary function to be optimized, this is
            % true.
          & \rot{general objective functions}
            %
            % If I can implement any new optimization technique to use, this is
            % true.
          & \rot{arbitrary optimizers (TODO: not clear)}
            %
            % If the framework could take advantage of when the gradient is
            % sparse, this is true.
          & \rot{sparse gradients}
            %
            % If the framework can handle categorical/discrete variables for
            % optimization, this is true.
          & \rot{categorical / discrete variables}
            %
            % If any type can be optimized, this is true.mention
            & \rot{various element types}
            %
            % If callback support is available.
          & \rot{callback functions} \\
        \cmidrule[1pt]{2-11}
        % It might be reasonable to say mlpack categorical support is only
        % partial, but I am not sure exactly where we draw the line.
        & \texttt{ensmallen}            & \CIRCLE & \CIRCLE & \CIRCLE & \CIRCLE & \CIRCLE & \CIRCLE & \CIRCLE & \CIRCLE & \CIRCLE\\
        % The Shogun toolbox has a fairly nice framework, but it doesn't support
        % sparse gradients or categorical features.  It also does not appear to
        % support constraints, arbitrary types, or callbacks.
        & Shogun \citep{sonnenburg2010shogun}             & \CIRCLE & - & \CIRCLE
& \CIRCLE & \CIRCLE & - & - & - & - \\
        % VW doesn't appear to have any framework whatsoever and the code is
        % awful, but it does support batches and categorical features.
        & VW \citep{Langford2007VW}      & - & - & \CIRCLE  & - & - & - &
\CIRCLE & - & - \\
        % TensorFlow has a few optimizers, but they are all SGD-related.  You
        % can write most objectives easily (but some very hard), and categorical
        % support might be possible but would not be easy.
        & TensorFlow \citep{tensorflow2015-whitepaper}        & \CIRCLE & -  & \CIRCLE  & \LEFTcircle & - &
\LEFTcircle & - & \LEFTcircle & - \\
        % PyTorch is increasingly popular these days.  It has Caffe integrated
        % into itself, but this refers to the actual PyTorch optimizer parts.
        & PyTorch \citep{NEURIPS2019_9015} & \CIRCLE & - & \CIRCLE & \LEFTcircle
& \LEFTcircle & - & - & \LEFTcircle & - \\
        % Caffe has a nice framework, but it's only for SGD-related optimizers.
        % I think I could write a new one, but it is not the easiest thing in
        % the world.
%        & Caffe \citep{jia2014caffe}           & \CIRCLE & -  & \CIRCLE & \LEFTcircle & \LEFTcircle
%& - & - & \LEFTcircle & \CIRCLE \\
        % Keras is restricted to neural networks and SGD-like optimizers.  I
        % don't know that it is possible to easily write a new optimizer.
        & Keras \citep{chollet2015keras}            & \CIRCLE & -  & \CIRCLE & \LEFTcircle & \LEFTcircle
& - & - & \LEFTcircle & \CIRCLE \\
        % sklearn has a few optimizer frameworks, but they are all in different
        % places and have somewhat different support.
        & scikit-learn \citep{pedregosa2011scikit}       & \LEFTcircle & - & \LEFTcircle  & \LEFTcircle & -
& - & - & \LEFTcircle & - \\
        % scipy has some nice optimizer framework but it does not support
        % batches or some of the more complex functionality.  And you can't
        % write your own.
        & SciPy \citep{2019arXiv190710121V}             & \CIRCLE & \CIRCLE  & -  &
\CIRCLE & - & - & - & \LEFTcircle & \CIRCLE \\
        % MATLAB is very similar to scipy.
        & MATLAB            & \CIRCLE & \CIRCLE & - &
\CIRCLE & - & - & - & \LEFTcircle & - \\
        % Optim.jl isn't the only Julia package for optimization, but it's the
        % one we compare against.
        % & \texttt{Optim.jl} \citep{mogensen2018optim}   &
        & \texttt{Optim.jl} (Mogensen et al.~2018) \nocite{mogensen2018optim}   &
\CIRCLE & \LEFTcircle & - & \CIRCLE & - & - & - & \CIRCLE & \CIRCLE \\
        \cmidrule[1pt]{2-11}
    \end{tabular}
\vspace*{-0.5em}
\caption{
Feature comparison:
\CIRCLE~= available,
\LEFTcircle~= partially available,
-~= not available.
% Feature comparison: \CIRCLE~= provides feature,
% \LEFTcircle~= partially provides feature, -~= does not provide feature.
%
%{\it unified framework} indicates if there is a form of generic/unified
%optimization framework; {\it constraints} and {\it separable functions /
%batches} indicate support for constrained functions and separable functions;
%{\it arbitrary functions} means arbitrary objective functions are easily
%implemented; {\it arbitrary optimizers} means arbitrary optimizers are easily
%implemented; {\it sparse gradient} indicates that the framework can natively
%take advantage of sparse gradients; {\it categorical} refers to if support
%for categorical features exists; {\it arbitrary types} mean that arbitrary types
%can be used for the parameters $x$;
%{\it callbacks} indicates that user-implementable callback support is available.
\label{tab:comparison}
\vspace{-1.5ex}
}
\end{table}

\section{Functionality}
\label{sec:overview}

The task of optimizing an objective function with {\tt ensmallen} is straightforward.
The type of objective function defines the implementation requirements.
Each type has a minimal set of methods that must be implemented,
which typically is only between one and four methods.
As an example,
to optimize an objective function $f(x)$ that is differentiable,
implementations of $f(x)$ and $f'(x)$ are required.
One of the optimizers for differentiable functions,
such as L-BFGS~\citep{liu1989limited},
can then be immediately employed.

Whenever possible, {\tt ensmallen} will automatically infer methods that are
not provided.  For instance, given a separable objective function $f(x) = \sum_i
f_i(x)$ where an implementation of $f_i(x)$ is provided (as well as the number
of such separable objectives), an implementation of $f(x)$ can be automatically
inferred.  This is done at compile-time, and so there is no additional runtime
overhead compared to a manual implementation.  C++ template metaprogramming
techniques~\citep{alexandrescu2001modern} are
internally used to automatically produce efficient code during compilation.

% In most cases, $f(x)$ has inherent attributes;
% for example, $f(x)$ might be differentiable.
% The internal framework in {\tt ensmallen} can optionally take advantage of such attributes.
% In the example of a differentiable function $f(x)$,
% the user can provide an implementation of the gradient $f'(x)$,
% which in turn allows a first-order optimizer to be used.
% To allow exploitation of such attributes, the optimizers are built to work with
% many types of objective functions.  {\tt ensmallen} thus has support for
% differentiable, partially differentiable, separable, categorical,
% and constrained functions.


%% ORIGINAL EXAMPLE CODE
%% 
% \begin{figure}[b!]
% \hrule
% \vspace{1ex}
% \centering
% \begin{minted}[fontsize=\scriptsize]{c++}
% 
% #include <ensmallen.hpp>
% 
% struct LinearRegressionFunction
% {
%   LinearRegressionFunction(const arma::mat& inX, const arma::vec& inY) : X(inX), y(inY) {}
% 
%   double Evaluate(const arma::mat& phi)  { return (X * phi - y).t() * (X * phi - y); }
%   void Gradient(const arma::mat& phi, arma::mat& grad)  { grad = 2 * X.t() * (X * phi - y); }
% 
%   const arma::mat& X; const arma::vec& y;
% };
% 
% int main()
% {
%   arma::mat X; arma::vec y;
%   // ... set the contents of X and y here ...
%   ens::LinearRegressionFunction f(X, y);
%   ens::L_BFGS optimizer; // create the optimizer with default parameters
%   arma::mat phi_best(X.n_rows, 1, arma::fill::randu);  // initial point (uniform random)
%   optimizer.Optimize(f, phi_best);
%   // at this point phi_best contains the best parameters
% }
% \end{minted}
% \hrule
% \vspace*{-0.5em}
% \caption{An example implementation of an objective function class for linear
% regression and usage of the L-BFGS optimizer in {\tt ensmallen}.
% %The online documentation for all ensmallen optimizers
% %is at \mbox{\url{https://ensmallen.org/docs.html}}.
% %The {\tt arma::mat} and {arma::vec} types are~
% %dense matrix and vector classes
% %from the Armadillo linear algebra library~\cite{sanderson2016armadillo},
% %with the corresponding online documentation at
% %\mbox{\url{http://arma.sf.net/docs.html}}
% %.
% }
% \label{fig:lr_function}
% \vspace*{-2em}
% \end{figure}


%% SIMPLIFIED EXAMPLE CODE
%% 
\begin{figure}[b!]
\hrule
\vspace{1ex}
\centering
\begin{minted}[fontsize=\scriptsize]{c++}

#include <ensmallen.hpp>

struct LinearRegressionFunction
{
  LinearRegressionFunction(const arma::mat& in_X, const arma::vec& in_Y) : X(in_X), y(in_Y) {}

  double Evaluate(const arma::mat& phi)  { arma::vec tmp = X * phi - y;  return arma::dot(tmp,tmp); }
  void Gradient(const arma::mat& phi, arma::mat& grad)  { grad = 2 * X.t() * (X * phi - y); }

  const arma::mat& X; const arma::vec& y;
};

int main() 
{
  arma::mat X; arma::vec y;
  // ... set the contents of X and y here ...
  ens::LinearRegressionFunction f(X, y);
  ens::L_BFGS optimizer; // create the optimizer with default parameters
  arma::mat phi_best(X.n_rows, 1, arma::fill::randu);  // initial point (uniform random)
  optimizer.Optimize(f, phi_best);
  // at this point phi_best contains the best parameters
}
\end{minted}
\hrule
\vspace*{-0.5em}
\caption{An example implementation of an objective function class for linear
regression and usage of the L-BFGS optimizer in {\tt ensmallen}.
%The online documentation for all ensmallen optimizers
%is at \mbox{\url{https://ensmallen.org/docs.html}}.
%The {\tt arma::mat} and {arma::vec} types are~
%dense matrix and vector classes
%from the Armadillo linear algebra library~\cite{sanderson2016armadillo},
%with the corresponding online documentation at
%\mbox{\url{http://arma.sf.net/docs.html}}
%.
}
\label{fig:lr_function}
\vspace*{-2em}
\end{figure}

% TODO: add basic details on the set of optimizers we implement

\section{Example Usage \& Empirical Comparison}
\label{sec:linreg_example}

For demonstration purposes, let us consider the problem of linear regression.
A matrix of predictors $\bm X \in \mathcal{R}^{n \times d}$
and a vector of responses $\bm y \in \mathcal{R}^n$ is given.
The task is to find the best linear model $\bm \Phi \in \mathcal{R}^d$,
which translates to finding
$\bm \Phi^* = \operatornamewithlimits{argmin}_{\bm\Phi} f(\bm \Phi)$ for
%$f(\bm \Phi) = \| \bm X \bm \Phi - \bm y \|^2 = (\bm X \bm \Phi - \bm y)^{\top} (\bm X \bm \Phi - \bm y).$
$f(\bm \Phi) = \| \bm X \bm \Phi - \bm y \|^2.$
From this we can derive the gradient
$f'(\bm \Phi) = 2 \bm X^{\top} (\bm X \bm \Phi - \bm y).$

% \footnote{Typically, in practice, solving a linear regression model can
% be done directly by taking the pseudoinverse.  But,
% this objective is easy to describe and useful for demonstration, so we
% use it here.}

To find $\bm \Phi^*$ using a differentiable optimizer,
we simply need to provide implementations of $f(\bm \Phi)$ and $f'(\bm \Phi)$.
For a differentiable function, {\tt ensmallen} requires only two methods:
{\tt Evaluate()} and {\tt Gradient()}.
The pre-built L-BFGS optimizer can then be used to find~$\bm \Phi^*$.
% we just need to provide an implementation of $f(\bm \Phi)$ and $f'(\bm \Phi)$
% as L-BFGS requires a differentiable objective function.
Figure~\ref{fig:lr_function} shows an example implementation.
%We hold {\tt X} and {\tt y} as members of the
%{\tt LinearRegressionFunction class},
%and {\tt theta} is used to represent $\bm \Phi$.
Via the use of the Armadillo library~\citep{sanderson2016armadillo},
the linear algebra expressions to implement the objective function and gradient
are compact and closely match natural mathematical notation.
%Equations~(\ref{eqn:obj_lr}) and~(\ref{eqn:grad_lr}).


% Details on how to implement and use each type of objective function~
% are omitted here for brevity.
% They can be found in the online documentation for {\tt ensmallen}
% at \mbox{\url{https://ensmallen.org/docs.html}}.
%%The details are subject to evolution over time.

%\section{Experiments}
%\label{sec:experiments}

%To show the efficiency of mathematical optimization with {\tt ensmallen}, we
%compare its performance with several other commonly used optimization
%frameworks, including some that use automatic differentiation.

%\subsection{Simple Optimizations and Overhead}

%For our first experiment, we aim to capture the overhead involved in various
%optimization toolkits.  In order to do this, we consider the simple and popular
%Rosenbrock function~\cite{Rosenbrock1960}:
%
%\begin{equation}
%f([x_1, x_2]) = 100 (x_2 - x_1^2)^2 + (1 - x_1^2).
%\end{equation}

%This objective function is useful for this task because the computational effort
%involved in computing $f(\cdot)$ is trivial.  Therefore, if we hold the number
%of iterations of each toolkit constant, then this will help us understand the
%overhead costs of each toolkit.  For the optimizer, we use simulated
%annealing~\cite{kirkpatrick1983optimization}, a gradient-free optimizer.
%Simulated annealing will call the objective function numerous times; for each
%simulation we limit the optimizer to 100K objective evaluations.

%The code used to run this simulation for {\tt ensmallen} (including the
%implementation of the Rosenbrock function) is given in
%Figure~\ref{fig:rosenbrock_run}.  Note that the {\tt RosenbrockFunction} is
%actually implemented in {\tt ensmallen}'s source code, in the directory {\tt
%include/ensmallen\_bits/problems/}.

% TODO: code snippet comparison for each language?

%We compare four frameworks for this task:
%
%\begin{itemize}
%\itemsep=-1ex
%  {\bf (i)} {\tt ensmallen},
%  {\bf (ii)} {\tt scipy.optimize.anneal} from SciPy 0.14.1~\cite{jones2014scipy},
%  {\bf (iii)} simulated annealing implementation in {\tt Optim.jl} with Julia 1.0.1~\cite{mogensen2018optim},
%and
%  {\bf (iv)} {\tt samin} in the {\tt optim} package for Octave~\cite{octave}.
%\end{itemize}

%While another option here might be {\tt simulannealbnd()}
%in the Global Optimization Toolkit for MATLAB,
%no license was available.
%We ran our code on a MacBook Pro i7 2018 with 16GB RAM running macOS 10.14 with clang 1000.10.44.2, Julia version 1.0.1, Python 2.7.15, and Octave 4.4.1.

%Our initial implementation for each toolkit, corresponding to the line
%``default'' in Table~\ref{tab:rosenbrock_results}, was as simple of an
%implementation as possible and included no tuning.  This reflects how a typical
%user might interact with a given framework.  Only Julia and {\tt ensmallen} are
%compiled, and thus are able to avoid the function pointer dereference for
%evaluating the Rosenbrock function and take advantage of inlining and related
%optimizations.  The overhead of both {\tt scipy} and {\tt samin} are quite
%large---{\tt ensmallen} is nearly three orders of magnitude faster for the same
%task.

%\begin{figure}[t!]
%\hrule
%\vspace{1ex}
%\begin{minted}[fontsize=\small]{c++}
%#include <ensmallen.hpp>
%
%struct RosenbrockFunction
%{
%  template<typename MatType>
%  static typename MatType::elem_type Evaluate(const MatType& x) const
%  {
%    return 100 * std::pow(x[1] - std::pow(x[0], 2), 2) + std::pow(1 - x[0], 2);
%  }
%};
%
%int main()
%{
%  arma::wall_clock clock;
%
%  RosenbrockFunction rf;
%  ens::ExponentialSchedule sched;
%  // A tolerance of 0.0 means the optimization will run for the maximum number of iterations.
%  ens::SA<> s(sched, 100000, 10000, 1000, 100, 0.0);
%
%  // Get the initial point of the optimization.
%  arma::mat parameters = rf.GetInitialPoint();
%
%  // Run the optimization and time it.
%  clock.tic();
%  s.Optimize(rf, parameters);
%  const double time = clock.toc();
%  std::cout << time << std::endl << "Result (optimal 1, 1): " << parameters.t();
%  return 0;
%}
%\end{minted}
%\hrule
%\vspace*{-0.5em}
%\caption{Code to use {\tt ensmallen} to optimize the Rosenbrock function using
%100K iterations of simulated annealing.}
%\label{fig:rosenbrock_run}
%\end{figure}

%\begin{table}[b!]
%\begin{center}
%\begin{tabular}{lcccc}
%\toprule
% & {\tt ensmallen} & {\tt scipy} & {\tt Optim.jl} & {\tt samin} \\
%\midrule
%default & {\bf 0.004s} & 1.069s & 0.021s & 3.173s \\
%tuned & & 0.574s & & 3.122s \\
%\bottomrule
%\end{tabular}
%\end{center}
%\vspace*{-0.5em}
%\caption{Runtimes for $100$K iterations of simulated annealing with various
%frameworks on the simple Rosenbrock function.  Julia code runs do not count
%compilation time.  The {\it tuned} row indicates that the code was manually
%modified for speed.}
%\label{tab:rosenbrock_results}
%\end{table}

% Actually Octave's JIT is apparently some kind of prototype joke and it doesn't
% even compile anymore.  So MEX was the only way...
%However, both Python and Octave have routes for acceleration,
%such as Numba~\cite{lam2015numba}, MEX bindings and JIT compilation.
%We hand-optimized the Rosenbrock implementation using Numba,
%which required significant modification of the
%underlying \texttt{anneal.anneal()} function.
%These techniques did produce some speedup,
%as shown in the second row of Table~\ref{tab:rosenbrock_results}.
%For Octave, a MEX binding did not produce a noticeable difference.
%We were unable to tune either \texttt{ensmallen} or
%\texttt{Optim.jl} to get any speedup,
%suggesting that novice users will easily be able
%to write efficient code in these cases.

%\subsection{Large-Scale Linear Regression Problems}


Table~\ref{tab:lbfgs} compares the performance
of {\tt ensmallen} against other optimization frameworks
for the task of optimizing linear regression parameters on various dataset sizes.
In each framework the provided L-BFGS optimizer is employed
and explicitly limited to $10$ iterations.
The data used are highly noisy random data with a slight linear pattern.
The runtimes are the average of 10 runs.

% allows us to share work across the objective function and gradient
% implementations, we denote this as {\tt ensmallen-1}\footnote{This results in a
% more efficient implementation; more details can be found in~\citet{ensmallen2020}.}


%We also implement a version with both \texttt{Evaluate()} and
%\texttt{Gradient()}: \texttt{ensmallen-2}.%  We also use automatic
%differentiation for Julia via the~
%\texttt{ForwardDiff.jl}~\cite{RevelsLubinPapamarkou2016} package
%and for Python via the \texttt{Autograd}~\cite{maclaurin2015autograd} package.~~
%For GNU Octave we use the \texttt{bfgsmin()} function.

% Note that the exact data is not relevant
% for the experiments here, only its size.
% We find that \texttt{ensmallen-1} and the non-work-sharing {\tt ensmallen-2}
% are the fastest approaches.
%Furthermore, the use of \texttt{EvaluateWithGradient()} yields
%non-negligible speedup over the \texttt{ensmallen-2} implementation with
%both the objective and gradient implemented.
% In addition, although
%the automatic differentiation support makes it easier for users to write their
%code (since they do not have to write an implementation of the gradient), we
%observe that the output of automatic differentiators is not as efficient,
%especially with \texttt{ForwardDiff.jl}.  We expect this effect to be
%more pronounced with increasingly complex objective functions.

% This is in part due to {\tt ensmallen}'s ability to produce efficient code via
% template metaprogramming and C++ features, while retaining an intuitive
% interface for implementation.

%% RESULTS WITH TWO ROWS FOR ENSMALLEN: ENSMALLEN-1 AND ENSMALLEN-2 VARIANTS
%% 
% \begin{table}[t!]
% {\small
% \centering
% %\begin{adjustbox}{scale={0.90}{0.90}}
% \begin{tabular}{lccccc}
% \toprule
% {\em algorithm} & $d$: 100, $n$: 1k & $d$: 100, $n$: 10k & $d$: 100, $n$:
% 100k & $d$: 1k, $n$: 100k \\
% \midrule
% \texttt{ensmallen-1} & {\bf 0.001s} & {\bf 0.009s} & {\bf 0.154s} & {\bf 2.215s} \\
% \texttt{ensmallen-2} & 0.002s & 0.016s & 0.182s & 2.522s \\
% % Dropped for space and awful performance
% %\texttt{Calculus.jl} & 0.172s & 0.960s & 27.428s & 2535.507s \\
% \texttt{Optim.jl} & 0.006s & 0.030s & 0.337s & 4.271s \\
% \texttt{scipy} & 0.003s & 0.017s & 0.202s & 2.729s \\
% \texttt{bfgsmin} (GNU Octave) & 0.071s & 0.859s & 23.220s & 2859.81s\\
% % It's possible to tune ForwardDiff.jl a bit, but it doesn't give significant
% % speedups to make it competitive and it really makes the code ugly.
% \texttt{ForwardDiff.jl} & 0.497s & 1.159s & 4.996s & 603.106s \\
% \texttt{autograd} (Python) & 0.007s & 0.026s & 0.210s & 2.673s \\
% \bottomrule
% \end{tabular}
% %\end{adjustbox}
% %\vspace*{0.25ex}
% \vspace*{-0.4em}
% \caption{
% Runtimes for the linear regression function on various dataset sizes,
% with $n$ indicating the number of samples,
% and $d$ indicating the dimensionality of each sample.
% %All Julia runs do not count compilation time.
% }
% \label{tab:lbfgs}
% }
% \vspace*{-2.2em}
% \end{table}

%% RESULTS WITH ONLY ONE ROW FOR ENSMALLEN
%% 
\begin{table}[t!]
{\small
\centering
%\begin{adjustbox}{scale={0.90}{0.90}}
\begin{tabular}{lccccc}
\toprule
{\em Framework} & $d$: 100, $n$: 1k & $d$: 100, $n$: 10k & $d$: 100, $n$:
100k & $d$: 1k, $n$: 100k \\
\midrule
\texttt{ensmallen} & {\bf 0.001s} & {\bf 0.009s} & {\bf 0.154s} & {\bf 2.215s} \\
% Dropped for space and awful performance
%\texttt{Calculus.jl} & 0.172s & 0.960s & 27.428s & 2535.507s \\
\texttt{Optim.jl} & 0.006s & 0.030s & 0.337s & 4.271s \\
\texttt{scipy} & 0.003s & 0.017s & 0.202s & 2.729s \\
\texttt{bfgsmin} (GNU Octave) & 0.071s & 0.859s & 23.220s & 2859.81s\\
% It's possible to tune ForwardDiff.jl a bit, but it doesn't give significant
% speedups to make it competitive and it really makes the code ugly.
\texttt{ForwardDiff.jl} & 0.497s & 1.159s & 4.996s & 603.106s \\
\texttt{autograd} (Python) & 0.007s & 0.026s & 0.210s & 2.673s \\
\bottomrule
\end{tabular}
%\end{adjustbox}
%\vspace*{0.25ex}
\vspace*{-0.4em}
\caption{
Runtimes for optimizing linear regression parameters on various dataset sizes,
where $n$ is the number of samples,
and $d$ is the dimensionality of each sample.
%All Julia runs do not count compilation time.
}
\label{tab:lbfgs}
}
\vspace*{-2.2em}
\end{table}


\section{Conclusion}
\label{sec:conclusion}

We have introduced the {\tt ensmallen} numerical optimization library,
which provides a flexible framework for mathematical optimization of user-supplied objective functions in C++.
Many types of objective functions are supported,
including general, differentiable, separable, constrained, and categorical.
A large set of pre-built optimizers is provided.
Implementation of a new objective function typically requires only one or two C++ functions.
Custom behavior during optimization can be specified via callback functions.
%The library internally uses template metaprogramming to automatically produce efficient code.
Further details, such as
internal use of template metaprogramming for automatic generation of efficient code, 
automatic function inference,
clean error reporting, and various approaches for obtaining efficiency
are discussed in the accompanying technical report~\citep{ensmallen2020}.
The underlying framework facilitates the implementation of new optimization techniques,
which can be contributed and incorporated into the library.


% This paper is a summarized version of a lengthy technical report with
% details~\citep{ensmallen2020}; readers should refer there or to the
% project homepage for more details.

%This report introduces and explains {\tt ensmallen}, a C++ mathematical
%optimization library that internally uses template metaprogramming to produce efficient
%code.  The library is flexible, with support for numerous types of objective
%functions, and many implemented optimizers.  It is easy to both implement
%objective functions to optimize with {\tt ensmallen}, and to write new
%optimizers for inclusion in the library.  {\tt ensmallen} has support for
%automatic function inference and callbacks.  Because this is done through
%template metaprogramming, there is no additional runtime overhead.
%Empirical results show that {\tt ensmallen} outperforms other toolkits~
%for similar tasks.

%Future work includes the implementation of additional optimizers
%and better support for various types of objective functions
%(such as better support for constrained functions).
The library has been successfuly used by open source projects
such as the {\it mlpack} machine learning toolkit~\citep{mlpack2018}.
The library uses the permissive BSD license~\citep{Laurent_2008},
with the development done in an open manner.
% at \mbox{\url{https://github.com/mlpack/ensmallen}}.
%Anyone is welcome to help with the effort and contribute code.
The source code and documentation and are freely available at \mbox{\url{https://ensmallen.org}}.


% Acknowledgements should go at the end, before appendices and references

\bibliography{refs}

\end{document}
