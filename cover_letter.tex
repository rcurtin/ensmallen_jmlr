\documentclass[twoside,11pt]{article}

% Any additional packages needed should be included after jmlr2e.
% Note that jmlr2e.sty includes epsfig, amssymb, natbib and graphicx,
% and defines many common macros, such as 'proof' and 'example'.
%
% It also sets the bibliographystyle to plainnat; for more information on
% natbib citation styles, see the natbib documentation, a copy of which
% is archived at http://www.jmlr.org/format/natbib.pdf

% Available options for package jmlr2e are:
%
%   - abbrvbib : use abbrvnat for the bibliography style
%   - nohyperref : do not load the hyperref package
%   - preprint : remove JMLR specific information from the template,
%         useful for example for posting to preprint servers.
%
% Example of using the package with custom options:
%
% \usepackage[abbrvbib, preprint]{jmlr2e}

\usepackage{jmlr2e}
\usepackage{minted}
\usepackage{textcomp}
\usepackage{graphicx}
\usepackage{amsmath}
\usepackage{bm}
\usepackage{booktabs}      % for \toprule in tables
\usepackage{multirow}      % for table multirow
\usepackage{tcolorbox}   % colorbox in minted
\usepackage{parcolumns}
\usepackage{adjustbox}
\usepackage{nicefrac}
\usepackage{tabularx}
\usepackage{array}
\usepackage{wasysym}
\usepackage{dcolumn}

% Definitions of handy macros can go here

\newcommand{\dataset}{{\cal D}}
\newcommand{\fracpartial}[2]{\frac{\partial #1}{\partial  #2}}
\newcolumntype{d}[1]{D{.}{.}{#1}}
\newcommand\mc[1]{\multicolumn{1}{c}{#1}}

\newcolumntype{R}[2]{%
  >{\adjustbox{angle=#1,lap=\width-(#2)}\bgroup}%
  l%
  <{\egroup}%
}
\newcommand*\rot{\multicolumn{1}{R{30}{2.0em}}}

\begin{document}

\noindent Dear Editors and Reviewers: \\

\medskip

\noindent This paper is a resubmission of JMLR manuscript 20-416.  We were
encouraged to resubmit a revised article after addressing the comments of the
reviewers; now that we have addressed each comment, we are ready to submit our
revision.  We hope that you find our updates satisfactory. \\

\noindent Below is a summary of changes:

\begin{itemize}
  \item Following the comments from Reviewer 1, we agree that Table 1 is
confusing, and we have simply removed it.  This allows more space for
experiments.

  \item To demonstrate the utility and efficiency of ensmallen on real-world
problems, we added an experiment comparing ensmallen to other toolkits to train
a logistic regression model on real-world datasets from the UCI dataset
repository.  Unfortunately, with only four pages, we were unable to figure out
how to include any more experiments in the given space.  Readers who are
interested in more information could refer to a few additional experiments in
the technical report that is referenced at the end of the conclusion.

  \item We also added PyTorch and TensorFlow to the list of competitors for our
experiments.

  \item We were concerned by the report of test failures by Reviewer 1 (thank
you for the report!).  Many tests in ensmallen are probabilistic---we choose a
random initial point, then ensure that the optimizer is able to reach somewhere
close to the expected minimum.  Convergence is not always guaranteed, so
sometimes we must run a handful of trials.  We overhauled our test
infrastructure such that we are now able to run our tests 1000 times with
different random seeds and no failures.  (Our changes can be seen in
\url{https://github.com/mlpack/ensmallen/pull/249}.)  This revamped test suite
is a part of the latest release, ensmallen 2.15.2.

  \item Following the suggestion of Reviewer 1, we added a simple test example
in the root of the repository ({\tt example.cpp}), and a link to it from the
ensmallen documentation at \url{https://ensmallen.org/docs.html}.  We agree that
this is a nice improvement; thanks to Reviewer 1 for suggesting this change.
\end{itemize}

\noindent We are always interested in more opportunities to improve our work,
and so we look forward to additional feedback. \\

\noindent Thank you!

\end{document}
