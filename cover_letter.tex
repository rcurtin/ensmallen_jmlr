\documentclass[twoside,11pt]{article}

% Any additional packages needed should be included after jmlr2e.
% Note that jmlr2e.sty includes epsfig, amssymb, natbib and graphicx,
% and defines many common macros, such as 'proof' and 'example'.
%
% It also sets the bibliographystyle to plainnat; for more information on
% natbib citation styles, see the natbib documentation, a copy of which
% is archived at http://www.jmlr.org/format/natbib.pdf

% Available options for package jmlr2e are:
%
%   - abbrvbib : use abbrvnat for the bibliography style
%   - nohyperref : do not load the hyperref package
%   - preprint : remove JMLR specific information from the template,
%         useful for example for posting to preprint servers.
%
% Example of using the package with custom options:
%
% \usepackage[abbrvbib, preprint]{jmlr2e}

\usepackage{jmlr2e}
\usepackage{minted}
\usepackage{textcomp}
\usepackage{graphicx}
\usepackage{amsmath}
\usepackage{bm}
\usepackage{booktabs}      % for \toprule in tables
\usepackage{multirow}      % for table multirow
\usepackage{tcolorbox}   % colorbox in minted
\usepackage{parcolumns}
\usepackage{adjustbox}
\usepackage{nicefrac}
\usepackage{tabularx}
\usepackage{array}
\usepackage{wasysym}
\usepackage{dcolumn}

% Definitions of handy macros can go here

\newcommand{\dataset}{{\cal D}}
\newcommand{\fracpartial}[2]{\frac{\partial #1}{\partial  #2}}
\newcolumntype{d}[1]{D{.}{.}{#1}}
\newcommand\mc[1]{\multicolumn{1}{c}{#1}}

\newcolumntype{R}[2]{%
  >{\adjustbox{angle=#1,lap=\width-(#2)}\bgroup}%
  l%
  <{\egroup}%
}
\newcommand*\rot{\multicolumn{1}{R{30}{2.0em}}}

\begin{document}

\noindent Dear Editors and Reviewers: \\

\medskip

\noindent This paper is a resubmission of JMLR manuscript 20-416.  We were
encouraged to resubmit a revised article after addressing the reviewers' comments.
Below is a summary of the changes:

\begin{itemize}
  \item Following the comments from Reviewer 1, we agree that Table 1 is
confusing. We have removed it.  This has opened up space to provide an extra experiment.

  \item To demonstrate the utility and efficiency of {\it ensmallen} on real-world
problems, we added an experiment comparing {\it ensmallen} to other toolkits in the context of training 
a logistic regression model on real-world datasets from the UCI dataset
repository.  While we would like to add even more experiments,
we are constrained by the four page limit for this manuscript.
However, additional experiments for interested readers are given
in the accompanying technical report, which is cited in the conclusion.

  \item As per the suggestion of Reviewer 1, we added PyTorch and TensorFlow
  for comparison in our experiments.

  \item We have addressed the test failures reported by Reviewer 1.
Many tests in {\it ensmallen} are probabilistic, where a random initial point is used,
followed by checking that an optimizer is able to reach somewhere
close to the expected minimum.  However, given the nature of many optimizers,
convergence to an expected point is not always guaranteed.
To address this, we overhauled our test infrastructure such that we are now able
to run our tests 1000 times with unique random seeds and no failures.
The revamped test suite is part of the latest release, ensmallen 2.15.2.
(The changes can be viewed in \url{https://github.com/mlpack/ensmallen/pull/249}).  

  \item Following the suggestion of Reviewer 1, we added a simple example program
in the root of the repository ({\tt example.cpp}), and added a link to it from the
online documentation at \url{https://ensmallen.org/docs.html}.
The example program is indeed very useful for newcomers.
\end{itemize}

\noindent We would like to thank the reviewers for their constructive feedback,
which has been very useful for improving both the software and the manuscript.

\end{document}
